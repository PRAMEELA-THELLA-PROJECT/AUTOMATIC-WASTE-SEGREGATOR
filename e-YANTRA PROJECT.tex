%%%%%%%%%%%%  Generated using docx2latex.com  %%%%%%%%%%%%%%

%%%%%%%%%%%%  v2.0.0-beta  %%%%%%%%%%%%%%

\documentclass[12pt]{article}
\usepackage{amsmath}
\usepackage{latexsym}
\usepackage{amsfonts}
\usepackage[normalem]{ulem}
\usepackage{soul}
\usepackage{array}
\usepackage{amssymb}
\usepackage{extarrows}
\usepackage{graphicx}
\usepackage[backend=biber,
style=numeric,
sorting=none,
isbn=false,
doi=false,
url=false,
]{biblatex}\addbibresource{bibliography.bib}

\usepackage{subfig}
\usepackage{wrapfig}
\usepackage{wasysym}
\usepackage{enumitem}
\usepackage{adjustbox}
\usepackage{ragged2e}
\usepackage[svgnames,table]{xcolor}
\usepackage{tikz}
\usepackage{longtable}
\usepackage{changepage}
\usepackage{setspace}
\usepackage{hhline}
\usepackage{multicol}
\usepackage{tabto}
\usepackage{float}
\usepackage{multirow}
\usepackage{makecell}
\usepackage{fancyhdr}
\usepackage{graphicx}
\usepackage[utf8]{inputenc}
\usepackage[toc,page]{appendix}
\usepackage[hidelinks]{hyperref}
\usetikzlibrary{shapes.symbols,shapes.geometric,shadows,arrows.meta}
\tikzset{>={Latex[width=1.5mm,length=2mm]}}
\usepackage{flowchart}\usepackage[paperheight=11.69in,paperwidth=8.27in,left=1.0in,right=1.0in,top=1.0in,bottom=1.0in,headheight=1in]{geometry}
\usepackage[utf8]{inputenc}
\usepackage[T1]{fontenc}
\TabPositions{0.5in,1.0in,1.5in,2.0in,2.5in,3.0in,3.5in,4.0in,4.5in,5.0in,5.5in,6.0in,}

\urlstyle{same}


 %%%%%%%%%%%%  Set Depths for Sections  %%%%%%%%%%%%%%

% 1) Section
% 1.1) SubSection
% 1.1.1) SubSubSection
% 1.1.1.1) Paragraph
% 1.1.1.1.1) Subparagraph


\setcounter{tocdepth}{5}
\setcounter{secnumdepth}{5}


 %%%%%%%%%%%%  Set Depths for Nested Lists created by \begin{enumerate}  %%%%%%%%%%%%%%


\setlistdepth{9}
\renewlist{enumerate}{enumerate}{9}
        \setlist[enumerate,1]{label=\arabic*)}
        \setlist[enumerate,2]{label=\alph*)}
        \setlist[enumerate,3]{label=(\roman*)}
        \setlist[enumerate,4]{label=(\arabic*)}
        \setlist[enumerate,5]{label=(\Alph*)}
        \setlist[enumerate,6]{label=(\Roman*)}
        \setlist[enumerate,7]{label=\arabic*}
        \setlist[enumerate,8]{label=\alph*}
        \setlist[enumerate,9]{label=\roman*}

\renewlist{itemize}{itemize}{9}
        \setlist[itemize]{label=$\cdot$}
        \setlist[itemize,1]{label=\textbullet}
        \setlist[itemize,2]{label=$\circ$}
        \setlist[itemize,3]{label=$\ast$}
        \setlist[itemize,4]{label=$\dagger$}
        \setlist[itemize,5]{label=$\triangleright$}
        \setlist[itemize,6]{label=$\bigstar$}
        \setlist[itemize,7]{label=$\blacklozenge$}
        \setlist[itemize,8]{label=$\prime$}



 %%%%%%%%%%%%  Header here  %%%%%%%%%%%%%%


\pagestyle{fancy}
\fancyhf{}
\chead{
\vspace{\baselineskip}
}
\renewcommand{\headrulewidth}{0pt}
\setlength{\topsep}{0pt}\setlength{\parindent}{0pt}

 %%%%%%%%%%%%  This sets linespacing (verticle gap between Lines) Default=1 %%%%%%%%%%%%%%


\renewcommand{\arraystretch}{1.3}


%%%%%%%%%%%%%%%%%%%% Document code starts here %%%%%%%%%%%%%%%%%%%%



\begin{document}
\tab \tab \tab {\fontsize{18pt}{21.6pt}\selectfont \textbf{IDEA PROPOSAL }\par}\par

\setlength{\parskip}{8.04pt}
\begin{Center}
{\fontsize{18pt}{21.6pt}\selectfont \textbf{e-Yantra Ideas Competition 2019-20}\par}
\end{Center}\par


\vspace{\baselineskip}
{\fontsize{14pt}{16.8pt}\selectfont \textbf{\uline{Project Name:}}\par}\tab \par

\textcolor[HTML]{0070C0}{Automatic waste segregator which separates recyclable, non-recyclable and organic waste materials}\par


\vspace{\baselineskip}
{\fontsize{14pt}{16.8pt}\selectfont \textbf{\uline{Introduction/Motivation:}}\par}\par

\textcolor[HTML]{0070C0}{Domestic and aquatic animals take hazardous waste materials and are suffering a lot. And most of them are losing their lives [1]. Not only animals, waste pickers are also suffering a lot [2]. So, to reduce the death rate of animals, we decided to separate the waste at early stage.}\par

\textcolor[HTML]{0070C0}{To overcome above mentioned problem, we design dustbin which can separate the wastes like organic, recyclable and non-recyclable by itself. }\par

{\fontsize{14pt}{16.8pt}\selectfont \textbf{\uline{Market Research / Literature Survey:}}\par}\par

\textcolor[HTML]{0070C0}{Many teams had done research on this problem and they came up with some solutions. }\par

\textcolor[HTML]{0070C0}{Vigneshramakrishnan , along with his team, made an attempt to segregate metals, organic and inorganic wastes. [3].They had used NILabVIEW software to do their project. }\par

\textcolor[HTML]{0070C0}{Priyanka , along with her team, designed Automatic waste segregator, to separate metals, dry and wet waste materials. They had used Arduino IDE sofware to complete their project. }\par

\textcolor[HTML]{0070C0}{Some had done a project to separate organic, recyclable and metals[4]. }\par

\textcolor[HTML]{0070C0}{But our idea is different from those,as we want to separate recyclable, non-recyclable and organic waste. Some metals are recyclable and some are not recyclable. We are separating that metals into recyclable and non-recyclable. This is advanced technology to those two projects. }\par


\vspace{\baselineskip}
{\fontsize{14pt}{16.8pt}\selectfont \textbf{\uline{Hardware requirements:}}\par}\par

\textcolor[HTML]{0070C0}{1) Arduino UNO}\par

\textcolor[HTML]{0070C0}{2) IR sensor}\par

\textcolor[HTML]{0070C0}{3) Inductive proximity sensor}\par

\textcolor[HTML]{0070C0}{4) Magnets}\par

\textcolor[HTML]{0070C0}{5) Geared motors}\par

\textcolor[HTML]{0070C0}{6) Motor driver IC}\par

\textcolor[HTML]{0070C0}{7) Power supply}\par

\textcolor[HTML]{0070C0}{8) Step-down transformer}\par

\textcolor[HTML]{0070C0}{9) Non-metal detector}\par


\vspace{\baselineskip}
{\fontsize{14pt}{16.8pt}\selectfont \textbf{\uline{Software requirements:}}\par}\par

\textcolor[HTML]{0070C0}{1) Arduino IDE}\par


\vspace{\baselineskip}
{\fontsize{14pt}{16.8pt}\selectfont \textbf{\uline{Implementation:}}\par}\par

\textcolor[HTML]{0070C0}{The dustbin is divided into three slots i.e., organic, recyclable and non-recyclable. When a material is dumped into the bin, it is detected by the IR sensor, which is placed at the top of the bin. Then the mechanism will begin. Now the material lwill move to inductive proximity sensor, if it is a metal, then it will be sensed by the sensor. Then the hall effect is applied to the sensed material and its conductance value is compared with the given reference values of different materials and after detecting the nature of the metal, it moves to either recyclable or non-recyclable bin. }\par

\textcolor[HTML]{0070C0}{If the material is not detected by the inductive proximity sensor, that means it may be either non-metal or organic material. Then it is moved through the non-metal detector and if it detects the non-metal then it is moved to recyclable bin. Otherwise, it is moved to the organic waste bin. }\par

{\fontsize{10pt}{12.0pt}\selectfont \textcolor[HTML]{0070C0}{Then the organic waste is used as natural fertilizers and the recyclable materials are moved to factories to recycle them. The non-recyclable materials can be converted to energy by using incineration process. }\par}\par

\textbf{\textcolor[HTML]{36363D}{IR Sensor:}}\par

\textcolor[HTML]{0070C0}{IR sensor is used to detect the presence of material. }\par

\textbf{\textcolor[HTML]{36363D}{Inductive proximity sensor:}}\par

\textcolor[HTML]{0070C0}{An inductive proximity sensor is a device that uses the principle of electromagnetic induction to detect the metals. }\par

\textbf{\textcolor[HTML]{36363D}{Hall effect:}}\par

\textcolor[HTML]{0070C0}{If the metal is detected then the hall effect is applied to it. Hall effect is the production of a potential difference across an electrical conductor when a magnetic field is applied in a direction perpendicular to that of the flow of current. The voltage value is compared with the values of all the metals and based on this, the nature of the metal is known. With respect this, the metal is moved to either recyclable or non-recyclable. }\par

\textbf{\textcolor[HTML]{36363D}{Non-metal detector:}}\par

\textcolor[HTML]{0070C0}{If the material is not a metal, then it is moved to the non-metal detector. The material is detected if it is a non-metal, then it is moved to the recyclable bin .If it is not detected by any sensor, that means it is organic waste. So it is moved to organic waste bin. }\par


\vspace{\baselineskip}

\vspace{\baselineskip}
\textbf{\textcolor[HTML]{36363D}{Flow chart:}}\par



%%%%%%%%%%%%%%%%%%%% Figure/Image No: 1 starts here %%%%%%%%%%%%%%%%%%%%
\begin{figure}[H]
\includegraphics[width=\linewidth]{EYANTRA (1).jpg}
\end{figure}


%%%%%%%%%%%%%%%%%%%% Figure/Image No: 1 Ends here %%%%%%%%%%%%%%%%%%%%

\par



\vspace{\baselineskip}

\vspace{\baselineskip}
{\fontsize{14pt}{16.8pt}\selectfont \textbf{\textcolor[HTML]{36363D}{\uline{Feasibility}:}}\par}\par

\textcolor[HTML]{0070C0}{The currently existing dustbins are having two slots to dumpwet and dry waste. This needs human involvement. Also, it is not suitable to separate recyclable and non-recyclable waste. So to make it suitable in separating waste into recyclable and non-recyclable, the automatic waste segregator is used. This doesn't need human involvement except the dumping of waste into it.}\par


\vspace{\baselineskip}

\vspace{\baselineskip}
{\fontsize{14pt}{16.8pt}\selectfont \textbf{\textcolor[HTML]{36363D}{\uline{Reference:}}}\par}\par

\textcolor[HTML]{0070C0}{1)https://www.pollutionsolutions-online.com/news/waste-management/21/breaking-news/animals-suffering-from-toxic-waste/23310}\par

\textcolor[HTML]{0070C0}{2) https://www.ncbi.nlm.nih.gov/pmc/articles/PMC6147112/$\#$ \_\_ffn\_sectitle}\par

\textcolor[HTML]{0070C0}{3) https://youtu.be/5GwQcKD7bJE}\par

\textcolor[HTML]{0070C0}{4) https://contest.techbriefs.com/2017/entries/sustainable-technologies/8483}\par


\printbibliography
\end{document}